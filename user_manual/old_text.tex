\section{Introduction}

This document describes a lock-in amplifier running on an Adafruit M4 Feather Express microcontroller board in combination with a PC running Python. It is part of the lab assignments of the course \textit{'Small Signals \& Detection'} of the University of Twente, Enschede, The Netherlands.

The microcontroller (MCU) board generates the reference signal \refXs{} and subsequently acquires the input signal \sigI{}. This data is sent over USB to a PC running the main graphical user interface in Python. The Python program shows the waveform graphs of the signals in real-time, performs the heterodyne mixing and filtering of the signals similar to a lock-in amplifier, and provides logging to disk.

\bigskip\noindent
Current specifications MCU:
\begin{itemize}
\item{Support for Atmel SAMD21 or SAMD51 chipsets}
\item{True analog-out waveform generator (\refXs{} between 0 to 3.3 V)}
\item{Differential analog-in data acquisition (\sigI{} between -3.3 to 3.3 V)}
\item{The DAC (digital-to-analog converter) and ADC (analog-to-digital converter) operate at 12 bits resolution at a sample rate of 5 kHz}
\item{Double-buffered binary-data transmission over USB to a PC running Python}
\end{itemize}

\bigskip\noindent
Current specifications Python:
\begin{itemize}
\item{Separate computing threads for real-time communication with the MCU, signal processing and graphing}
\item{Automatic detection of the MCU board by scanning over all COM ports}
\item{High-quality zero-phase distortion linear FIR filters}
\item{Optional OpenGL hardware-accelerated graphing}
\item{Data logging to file}
\end{itemize}
